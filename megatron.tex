\section{MegaTron}

To evaluate Spineless Traversal,
  we implement it for a simplified web layout algorithm
  implemented in the DSL of Figure~\ref{fig:dsl}
  and compiled by a compile we call MegaTron.
In this section, we first describe
  the layout modes implemented,
  then how we compile the DSL to a layout engine,
  and finally compare Spinless Traversal to Double Dirty Bit
  on a collection of real-world web page interactions.

\subsection{Web Layout Algorithm}
\label{sec:layout-impl}

To compare Spineless Traversal and Double Dirty Bit,
  we needed a web layout algorithm
  that was decoupled from its invalidation strategy.
Unfortunately, existing browser rendering engines
  are too complex and too tightly coupled
  to their current invalidation strategy,
  making it impossible to reuse them.
We instead implemented our own web layout algorithm
  based on the Cassius and MEDEA formalizations%
  ~\cite{cassius-1,cassius-2,yufeng-2}.
Naturally, this implementation is simplified
  compared to a real-world web browser
  and only implements a subset of features.
However, we took care to implement several features
  with complex invalidation behavior,
  including intrinsic width, flex-box layout, and positioning.
In total, our implementation is 700 lines of code
  in a custom DSL called Megatron,
  which compute approximately 50 fields for each layout node.
The remainder of this section describes
  several key layout features implemented in our layout algorithm.

\paragraph{Box model}.
Each layout node has an $x$, $y$, width and height field.
Typically the rectangle this represents contains
    all of the node's children, while sibling elements
    don't overlap.
In general, width is computed top-down
  (children's width depend on parent's width),
  height is computed bottom-up
  (parent's height is sum of children's heights)
  and $x$ and $y$ are computed in-order.
This forms long dependency chains between elements,
    where updating one element can cause many others
    to be transitively dirtied.
However, the \texttt{width},
    \texttt{min-width}, and \texttt{max-width} CSS properties,
    and similar for \texttt{height},
    can directly set width and height,
    breaking these dependency chains.
Our layout algorithm's support for these properties
    means this early-stopping behavior is tested.
That said, even the
  explicit \texttt{width} and \texttt{height} properties
  allow ``percentage values'' like \texttt{50\%},
  which are resolved relative to the parent and thus
  still create inter-node dependencies.
	
\paragraph{Line Breaking}
Line breaking lays out inline layout nodes (text)
  horizontally from left to right,
  until the edge of the parent box is reached,
  and which point the next layout node is placed
  at the left edge of its parent,
  one line below its previous sibling.
Line breaking requires handling control dependencies,
  where one layout nodes' width may move layout nodes
  from one line to another by changing line breaking.
Additionally, our layout algorithm
  allows different lines to have different height
  (based on the height of the largest layout node in the line),
  which introduces a field (line height)
  that is dependent on may different nodes (each word in the line).
This has a number of interesting effects for invalidation.
For example, adding a node node to a line
  may cause no change if it is not the tallest node in the line.
However, if the new node is now the tallest node,
  that may cause the line to become taller,
  which in turn can cause all other text on the page to move down,
  causing a lot of invalidation.

\paragraph{Display}
Nodes have a \texttt{display} property that changes
  whether they act line words (\texttt{inline})
  or paragraphs (\texttt{block}).
The \texttt{display} property can also be set to \texttt{none},
  in which case the layout nodes are not shown on the page
  and have almost no effect on layout.
Changing a layout node's \texttt{display} property
  between \texttt{block} and \texttt{none}
  is a common way to implement drop-down menus,
  pop-ups, and other elements that appear and disappear,
  such as the ``preview'' hover effects on Wikipedia.
Importantly,
  changes to \texttt{display: none} nodes
  should not invalidate any other nodes and should be fast.
On the other hand, a node switching
  from \texttt{display: none} to \texttt{display: block}
  effectively adds a subtree to the layout tree,
  which is handled specially in Spineless Traversal.
  
\paragraph{Position}
An element with \texttt{position: absolute}
  is manually assigned a position on the page
  by the web developer.
This is commonly used for popups, tooltips, and other effects
  that appear over other page content.
It is also common
  to change the manually-assigned $x$ and $y$ positions
  from JavaScript, such as to move a tooltip away from the cursor.
Layout nodes with absolute positioning do not affect
  the position of other layout nodes outside their subtree,
  and handling changes to $x$ and $y$ positions quickly
  is essential.

\paragraph{Intrinsic sizes}
In a number of cases,
  a layout node's width or height is computed from
  its ``intrinsic'' size,
  which effectively measures how large it would need to be
  to contain all its text without line breaks.
For example, absolutely-positioned elements
  compute their width and height this way by default.
Importantly, intrinsic widths are computed bottom-up,
  but then used in the top-down width computation
  and then the bottom-up height computation.
This means intrinsic sizes require the use of
  multiple layout phases
  and require Spineless Traversal to support such.

\paragraph{Flex-box}
Flex-box layout is the most complex feature
  our layout algorithm supports.
In flex-box layout there is flex container element
  whose children are flex items.
The width/height of flex items depends on
  the intrinsic sizes of the other flex items and
  the actual size of the flex container.
Properties like \texttt{flex-grow} and \texttt{flex-shrink}
  determine how the intrinsic sizes of the flex items
  are adjusted to compensate for either extra or not enough
  available space in the flex container.
The \texttt{max-} and \texttt{min-width}/\texttt{height} properties
  can also cap the growth/shrinkage of individual flex items.
In all, our implementation of flex-box layout
  uses 9 of intermediate fields
  and requires 2 passes to compute all of them.

\paragraph{Miscellaneous}
We also implemented a variety of miscellaneous features,
  including automatic sizing of images and video,
  manual line breaks with the \texttt{<br>} element,
  and conditional elements like \texttt{<noscript>}
  (which are only rendered if JavaScript is disabled,
  not the case in our tests).
We also had to add a special case for \texttt{<svg>} elements,
  whose children describe drawing commands
  that do not participate in layout.
Finally, a variety of minor tweaks
  like the \texttt{width} and \texttt{height} HTML attributes
  (which behave slightly differently from the CSS properties)
  were also implemented.

\subsection{Compiler}
\label{sec-compiler}

\todo{Talk about spineless tagless}

This layout algorithm is implemented in the Megatron DSL
  and then compiled to C++ to maximize performance.
Passes are compiled to three C++ functions:
  the pass itself,
  a function that performs just the pre-order assignments
  and a function that performs just the post-order assignments.
The compiler type-checks all fields, attributes, and properties
  using Hindley-Milner type inference~\cite{HM}
  and uses appropriate unboxed C++ member variables
  to store the relevant values.
Importantly, this means a single node and all its fields
  are contiguous in memory
  (as it would be a real web browser)
  with a minimum of pointers
  (beyond the standard parent/first/last/next/previous pointers
  to other layout nodes)
  and that field access is compiled to a memory offset.
All string values (like the keyword values for \texttt{display})
  are interned and represented in C++ as
  a single \texttt{enum} type,
  meaning that no string allocation or comparison
  is performed at runtime.
Discriminated unions are used for fields with units.
The bottom line is that the compiled code is
  long but readable, idiomatic, and performant C++ code
  with no allocation, hash, or string operations.
This is critical, because it means that,
  like in a real web browser,
  computing fields is very fast
  and the cache misses from finding dirty fields
  are a measurable fraction of the runtime.

The compiler also synthesizes all
  dirty bit propagation code.
Specifically, the compiler examines
  each assignment in each pass,
  determines which fields on which neighbor nodes are read
  to compute each field,
  and computes the inverse neighbor relation.
Every field assignment
  compares the new and old value in the field
  and sets dirty bits only if the new and old value differ.
The actual invalidation traversal algorithm
  (Double Dirty Bit or Spineless)
  is encapsulated in an global value
  that provides a method to dirty a node/field
  and an iterator over all dirty nodes.
The incremental layout entry point function
  iterates over all dirty nodes
  and calls the relevant pass functions.
The compiler also generates functions to modify the tree,
  either by inserting and deleting tree nodes
  or by changing an attribute or property.
These generated functions dirty the appropriate nodes and fields,
  including using the special case for subtree insertion
  for Spineless Traversal.

To ensure our implementation is correct,
  the compiler can also generate a from-scratch layout function
  which does not use dirty bits at all
  and instead recomputes the entire layout from scratch.
This was extremely valuable during development
  and gives us confidence that our invalidation algorithm is correct.
