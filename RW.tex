\section{Related Work}

\paragraph{Incremental Computation}
Incremental computation---%
  speeding up computations by reusing previously-computed results---%
  is a long-studied topic in computer science broadly~\cite{memo}
  and programming language theory in particular;
  \citet{IC-Survey} and \citet{IC-bib}
  give thorough surveys of the fielld.
The recent Self-Adjusting Computation (SAC)~\cite{SAC} framework
  proposes incrementalizing arbitrary computations,
  including a cost semantics~\cite{SACCost},
  optimizations for data structure operations~\cite{SACTrace},
  and opportunities for parallelization~\cite{PSAC}.
The Adapton framework~\cite{Adapton}
  aims at \emph{demand-driven} incremental computation,
  and allows manually-specified annotations~\citet{AdaptonName}
  for greater reuse.
While this prior work focuses on general-purpose computations,
  Spineless Traversal is focused on a particularly-critical application
  of incremental computation: web browsers.
This application-specific focus has precedent:
  \citet{ICC} speed up memoization for functional programs over lists
  using ``chunky decomposition'',
  while differential dataflow databases~\cite{DDF}
  incrementalize Relational Algebra computations
  and have achieved some prominence in industry.

\citet{TR1} is the earliest work
  on incremental evaluation of attribute grammar,
  motivated by syntax-directed editors;
Later work~\cite{TR2} allows reference to non-neighbor attributes.
However, these early papers require recomputation
  immediately after every tree change,
  whereas web browsers, our target application,
  batch updates to perform incremental computation
  only once per frame.
The standard in web browsers is instead
  the Double Dirty Bit algorithm,
  implemented in all existing web rendering engines.
\citet{tali-garseil} describes the Double Dirty Bit algorithm
  in industry publication \texttt{web.dev},
  and it also features in textbooks~\cite{wbe}.
  
The formal methods community has put significant effort
  into formalizing web page layout.
\citet{meyerovich-1} proposed using attribute grammars,
  similar to the DSL in Figure~\ref{fig:dsl},
  for formalizing web-like layout rules.
Later work~\cite{yufeng-1} proposes
  synthesizing schedules from the attribute grammar rules,
  including proposals~\cite{meyerovich-2,meyerovich-3}
  to use parallel schedules to further improve layout performance.
Parallel schedules could be combined with Spineless Traversal
  to reduce latency even further.
The Cassius project~\cite{cassius-1}
  later formalized a significant fragment of CSS~2.1
  using an attribute-grammar-like formalism.
Our layout implementation is based on Cassius.
Later work~\cite{cassius-2} also proposed
  using the Cassius formalism to verify web page layouts,
  including in a modular way~\cite{cassius-3}.
However, none of these works investigated incremental layout.
By contrast, the MEDEA project~\cite{yufeng-2}
  proposed synthesizing incremental layout algorithms
  by automatically synthesizing dirty bit propogation code.
Our work extends MEDEA by exploring
  optimized incremental traversal algorithms.
