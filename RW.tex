\section{Related Work}

\paragraph{Incremental Computation}
Speeding up computations
  by reusing previously-computed results
  is a long-studied topic in computer science broadly~\cite{memo}
  and programming language theory in particular;
  \citet{IC-Survey} and \citet{IC-bib}
  give thorough surveys of the field.
The recent Self-Adjusting Computation (SAC)~\cite{SAC} framework
  proposes incrementalizing arbitrary computations,
  including a cost semantics~\cite{SACCost},
  optimizations for data structure operations~\cite{SACTrace},
  and opportunities for parallelization~\cite{PSAC}.
The Adapton framework~\cite{Adapton}
  aims at \emph{demand-driven} incremental computation,
  and allows manually-specified annotations~\cite{AdaptonName}
  for greater reuse.
While this prior work focuses on general-purpose computations,
  Spineless Traversal is focused on a particularly critical application: web browser layout.
This application-specific focus has precedent:
  \citet{ICC} speed up memoization for functional programs over lists
  using ``chunky decomposition'',
  while differential dataflow~\cite{DDF},
  which incrementalizes relational algebra in databases,
  is prominent in industry.

\citet{TR1} wrote the earliest work
  on incremental evaluation of attribute grammars,
  motivated by syntax-directed editors;
  later work~\cite{TR2} allows references
  to non-neighbor attributes.
However, these early papers require recomputation
  immediately after every tree change,
  whereas web browsers, our target application,
  batch multiple updates
  and perform layout only once per frame.
The standard in web browsers is instead
  the Double Dirty Bit algorithm,
  described in industry publications~\cite{tali-garseil}
  and textbooks~\cite{wbe}.
  
The formal methods community has put significant effort
  into formalizing web page layout.
\citet{meyerovich-1} proposed using attribute grammars,
  similar to the DSL in \Cref{fig:dsl},
  for formalizing web-like layout rules.
Later work~\cite{yufeng-1} proposes
  synthesizing schedules from the attribute grammar rules,
  including proposals~\cite{meyerovich-2,meyerovich-3}
  to use parallel schedules to further improve layout performance,
  though these proposals have not proven practical~\cite{servo-no-parallel}.
The Cassius project~\cite{cassius-1}
  formalizes a significant fragment of CSS~2.1
  using an attribute-grammar-like formalism.
Our layout implementation is based on Cassius.
Later work also proposed
  using the Cassius formalism to verify web page layouts~\cite{cassius-2},
  including in a custom proof assistant~\cite{cassius-3}.
However, none of these works investigate incremental layout.
By contrast, the \textsc{Medea} project~\cite{yufeng-2}
  proposed synthesizing incremental layout algorithms
  by automatically synthesizing dirty bit propagation code.
Our work extends \textsc{Medea} by exploring
  optimized incremental traversal algorithms.
