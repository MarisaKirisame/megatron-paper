\documentclass[format=acmsmall, review=false, screen=true]{acmart}

\usepackage{booktabs} % For formal tables

\usepackage[ruled]{algorithm2e} % For algorithms
\renewcommand{\algorithmcfname}{ALGORITHM}
\SetAlFnt{\small}
\SetAlCapFnt{\small}
\SetAlCapNameFnt{\small}
\SetAlCapHSkip{0pt}
\IncMargin{-\parindent}


% Metadata Information
\acmJournal{TWEB}
\acmVolume{9}
\acmNumber{4}
\acmArticle{39}
\acmYear{2010}
\acmMonth{3}
\copyrightyear{2009}
%\acmArticleSeq{9}

% Copyright
%\setcopyright{acmcopyright}
\setcopyright{acmlicensed}
%\setcopyright{rightsretained}
%\setcopyright{usgov}
%\setcopyright{usgovmixed}
%\setcopyright{cagov}
%\setcopyright{cagovmixed}

% DOI
\acmDOI{0000001.0000001}

% Paper history
\received{February 2007}
\received[revised]{March 2009}
\received[accepted]{June 2009}

\usepackage{xcolor}
\newcommand\todo[1]{\textcolor{red}{#1}}

% Document starts
\begin{document}
% Title portion. Note the short title for running heads 
\title[Spineless]{Spineless Traversal for web layout}  
\author{Gang Zhou}
\orcid{1234-5678-9012-3456}
\affiliation{%
  \institution{College of William and Mary}
  \streetaddress{104 Jamestown Rd}
  \city{Williamsburg}
  \state{VA}
  \postcode{23185}
  \country{USA}}
\author{Yafeng Wu}
\affiliation{%
  \institution{University of Virginia}
  \department{School of Engineering}
  \city{Charlottesville}
  \state{VA}
  \postcode{22903}
  \country{USA}
}
\author{Ting Yan}
\affiliation{%
  \institution{Eaton Innovation Center}
  \city{Prague}
  \country{Czech Republic}}
\author{Tian He}
\affiliation{%
  \institution{University of Minnesota}
  \country{USA}}
\author{Chengdu Huang}
\author{John A. Stankovic}
\author{Tarek F. Abdelzaher}
\affiliation{%
  \institution{University of Virginia}
  \department{School of Engineering}
  \city{Charlottesville}
  \state{VA}
  \postcode{22903}
  \country{USA}
}

\begin{abstract}
Multifrequency media access control has been well understood in
general wireless ad hoc networks, while in wireless sensor networks,
researchers still focus on single frequency solutions. In wireless
sensor networks, each device is typically equipped with a single
radio transceiver and applications adopt much smaller packet sizes
compared to those in general wireless ad hoc networks. Hence, the
multifrequency MAC protocols proposed for general wireless ad hoc
networks are not suitable for wireless sensor network applications,
which we further demonstrate through our simulation experiments. In
this article, we propose MMSN, which takes advantage of
multifrequency availability while, at the same time, takes into
consideration the restrictions of wireless sensor networks. Through
extensive experiments, MMSN exhibits the prominent ability to utilize
parallel transmissions among neighboring nodes. 
\end{abstract}


%
% The code below should be generated by the tool at
% http://dl.acm.org/ccs.cfm
% Please copy and paste the code instead of the example below. 
%
\begin{CCSXML}
<ccs2012>
 <concept>
  <concept_id>10010520.10010553.10010562</concept_id>
  <concept_desc>Computer systems organization~Embedded systems</concept_desc>
  <concept_significance>500</concept_significance>
 </concept>
 <concept>
  <concept_id>10010520.10010575.10010755</concept_id>
  <concept_desc>Computer systems organization~Redundancy</concept_desc>
  <concept_significance>300</concept_significance>
 </concept>
 <concept>
  <concept_id>10010520.10010553.10010554</concept_id>
  <concept_desc>Computer systems organization~Robotics</concept_desc>
  <concept_significance>100</concept_significance>
 </concept>
 <concept>
  <concept_id>10003033.10003083.10003095</concept_id>
  <concept_desc>Networks~Network reliability</concept_desc>
  <concept_significance>100</concept_significance>
 </concept>
</ccs2012>  
\end{CCSXML}

\ccsdesc[500]{Computer systems organization~Embedded systems}
\ccsdesc[300]{Computer systems organization~Redundancy}
\ccsdesc{Computer systems organization~Robotics}
\ccsdesc[100]{Networks~Network reliability}

%
% End generated code
%

% We no longer use \terms command
%\terms{Design, Algorithms, Performance}

\keywords{Wireless sensor networks, media access control,
multi-channel, radio interference, time synchronization}


\thanks{This work is supported by the National Science Foundation,
  under grant CNS-0435060, grant CCR-0325197 and grant EN-CS-0329609.

  Author's addresses: G. Zhou, Computer Science Department, College of
  William and Mary; Y. Wu {and} J. A. Stankovic, Computer Science
  Department, University of Virginia; T. Yan, Eaton Innovation Center;
  T. He, Computer Science Department, University of Minnesota; C.
  Huang, Google; T. F. Abdelzaher, (Current address) NASA Ames
  Research Center, Moffett Field, California 94035.}


\maketitle

% The default list of authors is too long for headers}
\renewcommand{\shortauthors}{G. Zhou et al.}

\section{Technical Section}

\subsection{Theory(4)}

\subsubsection{Attribute Grammar}
We assumed that web page layout can be implemented in attribute grammar, and the attribute grammar have been scheduled into multiple 'there and back again' pass.

That is, web page layout call a sequence of mutating function, where each function:
\begin{itemize}
	\item compute value for field x0, x1, x2...
	\item recursively invoke itself for each of the children, left to right.
	\item compute value for field y0, y1, y2...
\end{itemize}

Note that in the above formalization, both control flow and data flow is fixed and tied to the tree structure. We argue that this is a reasonable restriction to web layout as we are able to implement a wide variety of complex feature under such assumption.

We hypothesize that this is because web layout are \b{local}: the position of a dom node is highly related (i want to say determined but that sounds a bit too strong) to its relative. \todo{Hard to write this section without referring to DSL}

\subsubsection{Dirty Bit}
The standard method for incremental web layout is the dirty bit algorithm. \\
For each node, for each basic block BB, the algorithm keep a boolean variable, BB\_dirtied, Similarly, for each procedure P, the algorithm keep a boolean variable, P\_recursive\_dirtied, both initially set at false.

A field is dirtied, when one of it's dependency change it's value. A basic block is dirtied when one of it's field is dirtied. A procedure P is recursively\_dirtied, when one of it's corresponding basic block is dirtied, or when one of it's children's P is also recursively\_dirtied.

To recalculate, the incremental algorithm traverse the tree for each procedure, there-and-back-again style, similar to how the value is initially calculated. However, only basic block with BB\_dirtied=true will be re-executed, and once re-executed, said variable will be set to false. Additionally, once a node does not have the corresponding P\_recursive\_dirtied set, there is no need to recurse further.

\subsection{PQ(5)}
OM
Init/Mark/Recalculate
PQ Impl
\subsection{Impl(3)}
Below we give a syntax and semantics of the DSL megatron, and explain the design rational.

M(ain) := P\_N()... \\
P(roc)\_N := self.BB\_X(); children.P\_N(); self.BB\_Y() \\
BB(BasicBlock)\_X := self.N <- T... \\
V(ar) := unique symbols \\
F(unction) := a predefined set of primitive functions \\
P(ath) := self | prev | next | parent | first | last \\
T(erm) := V | if T then T else T | F(T...) | P.V

HTML Features

\begin{itemize}
	\item visibility (display)
	\item position (static vs absolute)
	\item line breaking
	\item flex
	\item box model
	\item intrinsic width/height
	\item fixed width/height
	\item min/max width/height
\end{itemize}

\todo{We also have some features that is too small for a bullet point (e.g. image with width and not height/vice versa), which should not be talked about in detail, but i still think we should brief over. how should i structure it?}

\end{document}
