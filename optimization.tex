\section{Optimization}
The Double Dirty Bit algorithm uses minimal computation (just set a bit!)
  but is extremely cache-unfriendly (accesses the spine + 1),
  while the spineless traversal is computationally expensive
  (priority queue and order maintenance)
  but cache-friendly (accesses only necessary nodes).
For spineless traversal to be faster,
  the computationally expensive steps in spineless traversal
  have to be meticulously optimization
  to avoid allocation, reduce memory traffic,
  and optimize branch mispredictions.

\subsection{Subtree Insertion}
\label{sec:tree-insertion}
Bulk insertions into the layout tree are common in our data set.
This typically seems to be the result of lazy loading:
  a ``shell'' web page loads first, and shows a loading indicator;
  then the ``content'' loads and is inserted into the page
  as a single large subtree.
This can occur in several stages, with a ``shell''
  first inserting ``subshells'' which in turn load subcomponents themselves.
Handling this requires
  efficient bulk insertion of a large subtree in one step.%
\footnote{
  Just enqueueing all fields of all nodes in the subtree
    is not a good idea---it will create a large queue with slow queue operations.
  In our experiments, this often took the queue
    from tens to thousands of members, leading to a small integer slowdown.
}

Our solution treats inserted subtrees as a single unit in the priority queue.
In other words, queue elements now refer to either a single node
  or to the whole subtree rooted at that node;
  the order maintenance order for either node is the same.
At allocation time, new layout nodes have all their fields dirty;
  thus, a queue cannot contain both a single-node and subtree object
  for the same node.
When a subtree object is dequeued from the priority queue,
  the current layout pass is performed on the whole subtree,
  creating all necessary order maintenance nodes in the correct order.
Throughout this pass, no nodes are enqueued or dequeued
  saving queue traffic and shrinking the size of the queue.
Note that any nodes that refer to the newly-inserted node
  do so through paths like \textsf{prev} or \textsf{first};
  these fields are dirtied when the subtree is inserted
  so do not need to be dirtied again
  when the subtree layout pass is done.

\subsection{Pointer Compression and Custom Allocator}
Order maintenance objects have to be allocated every time
  new layout nodes are inserted;
  optimizing that allocation is essential.
We use a hand-written pool allocator for order maintenance objects;
  in fact, we use separate pools for high- and low-level objects
  to enhance locality.
Since allocations are always for the same size,
  our custom allocator is significantly simpler than the system allocator.

Our pooled allocator, \texttt{OMPool},
  is shown in Figure~\ref{fig:allocator}.
It is paramaterized by \emph{two} types:
  the type of allocated object \texttt{T}
  and the index type \texttt{P} for pointers to allocated objects.
Crucially,
  \texttt{malloc} and \texttt{free} return and consume
  the pointer type \texttt{P} instead of raw pointers.
By making \texttt{P} a smaller integer type,
  like \texttt{uint32} or \texttt{uint16}, this not only saves memory but also increases
  the number of order-maintenance objects per cache line,
  which in turn improves throughput.
In our implementation, we use 32-bit pointers;
  this limits web pages to a few billion elements,
  a size far beyond what any browser can handle,
  but conveniently makes the total size
  of an order maintenance object 128 bits.
Exactly two order maintenance objects
  then fit in each cache line,
  with no order maintenance objects split across two.
By contrast, an experiment with 16-bit pointers
  actually increased runtime---because, we suspect,
  it lead to order maintenance objects
  splitting across two cache lines,
  which introduced additional stalls.

The actual implementation of \texttt{OMPool} is standard;
  it stores a \texttt{pool} of memory as a \texttt{std::vector},
  in which we ensure sufficient capacity at startup.
Freed elements are placed in a separate \texttt{freed} vector,
  which is preferentially drawn from by \texttt{malloc}.
Because the objects are all the same size,
  there is no fragmentation and \texttt{malloc}/\texttt{free}
  are nearly instantaneous.
Moreover, since spineless traversal
  creates order maintenance objects in order,
  allocation patterns are extremely favorable,
  with nearby nodes placed nearby in memory.

\begin{figure}
\begin{verbatim}

template<
  typename T,             // Type of allocated object
  typename P=uint32_t     // Integer type for "pointers"
> struct OMPool {
  std::vector<T> pool;    // Fast allocation, maximize cache usage

  std::vector<P> freed;   // Rapid reuse minimizes cache churn
  
  // Implementation is straightforward
  T* addressof(P p) { return &(pool[p]); }
  P malloc();
  void free(P p) { freed.push_back(p); }
}
\end{verbatim}
\caption{A pooling allocator that focuses on reducing cache misses.}
\label{fig:allocator}
\end{figure}

\subsection{Branchless Order Maintenance Comparison}
Priority queue pushes and pops spend basically all of their time
  comparing order maintenance objects.
Order maintenance objects are small and,
  thanks to our allocator, local in memory,
  making cache misses rare when accessing them.
However, order maintenance object comparison has two cases
  (same or different second-level lists)
  and the bottleneck ends up being the pipeline stall
  induced by the conditional.
The branch predictor does not help much,
  because (thanks to the heap) the comparison is unpredictable.
We therefore implemented a branchless comparison function,
  relying on conditional move instructions instead;
  in a microbenchmark, this reduces comparison time
  to 5 cycles; Figure~\cite{fig:compare} shows the
  assembly implementation.


\begin{figure}
\begin{verbatim}
bool operator<(const _l2_node &l, const _l2_node &r) {
  Label lpl = l.parent->label, rpl = r.parent->label;
  Label ll = l.label, rl = r.label;
  uint64_t result;

  asm volatile(
    "xor %%rbx, %%rbx\n"
    "cmp %1, %2\n"
    "seta %%bl\n"
    "xor %%rax, %%rax\n"
    "cmp %3, %4\n"
    "seta %%al\n"
    "cmove %%rbx, %%rax\n"  
    : "=&a"(result)
    : "r"(ll), "r"(rl), "r"(lpl), "r"(rpl));

  return result;
}
\end{verbatim}
\caption{
  The branchless comparison code is 7 instructions long,
    but the two \texttt{xor} instructions
    clear a register and are thus handled by the register renamer;
    the result is a five-cycle comparison
    for order-maintenance objects.
}
\label{fig:compare}
\end{figure}

\subsection{Dependency Deduplication}

When synthesizing the dirty bit propagation code,
  we make sure to only dirty any given field once per field computation.
For example, if a field $N.V$ is used twice in an expression,
  or if two different paths $\mathsf{first}.V$ and $\mathsf{last}.V$
  reverse to the same field,
  the field is only dirtied once.
This deduplication is especially challenging in the case
  of $N?$ expressions,
  since whether or not a field is dirtied can depend on whether an element
  is the first/last/other child of its parent.
Moreover, the dirty bit is checked before pushing to the priority queue;
  this means a node field only appears in the priority queue once,
  which keeps the queue small.

\subsection{Field packing}

To further shrink the size of the priority queue,
  we use a single dirty bit to cover multiple co-computed fields.
Instead of a dirty bit per field,
  we use two dirty bits per pass---%
  one for the pre-order-computed fields
  and one for post-order-computed fields.
This further reduces the number of unique dirty bits
  set during the dirty bit propagation.
Field packing in this way is a common optimization in real browsers;
  the trade-off is that it reduces the complexity
  (and thus cache misses in) invalidation traversal
  at the cost of possibly more field recomputations
  (since the browser must now recompute \emph{all} covered fields
  when a dirty bit is set)l
  the trade-off appears to be worth it.
The priority queue now likewise store objects
  that name a pass (and a pre/post location)
  instead of a specific field,
  and order maintenance objects likewise correspond
  to passes (plus pre/post) instead of to individual fields.
This reduces the number of OM objects needed
  and reduces the size of the priority queue.
More generally, one can think of the priority queue
  as storing dirty bits themselves,
  however those are assigned by the application in question.
Spineless traversal can thus be applied
  to finer- or coarser-grained field packing,
  though it also benefits more from coarser-grained packing,
  which makes the priority queue smaller and speeds up operations on it.

\subsection{Field representation}

Recall that priority queue entries
  refer to either a subtree or a node,
  and also name a phase plus pre/post location.
We combine the subtree-or-node bit with the phase and pre/post location
  into a simple enum, which we store in a byte.
We use the enum as an index into a jump table,
  while allows us to efficiently execute the field recomputation
  that corresponds to a single priority queue entry.

\subsection{Attempted, Failed Optimizations}

We also attempted a number of further optimizations
  that did not improve performance.
Ultimately, low-level cycle counting is an empirical art,
  and we don't have clear explanations for why these failed.
\begin{itemize}
\item A hybrid between Double Dirty Bit and spineless traversal,
  using a summary bit for subtree dirty bit propagation
  but the priority queue for more distant jumps.
  We were unable to make switching between the two modes efficient enough
  to be competitive.
\item Use a splay tree for the priority queue to improve locality.
  Resulted in a slowdown.
\item Use a red-black tree for the priority queue.
  Resulted in a slowdown. The classic min-heap is best.
\item Use a 1-based array representation for binary heap
  to cut instructions when finding the parent/children.
  Resulted in a slowdown.
\item Cut \texttt{OMPool} pointers to 16 bits, and use 16-bit OM labels.
  Did not see a speedup compared to 32 bits.
\item Pointer tagging to move boolean bit into pointers to reduce object size.
\end{itemize}