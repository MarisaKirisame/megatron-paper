\section{Optimization}
\label{sec:opt}

By avoiding auxiliary node accesses,
  Spineless Traversal can be
  dramatically faster than Double Dirty Bit
  in some cases.
However, to make Spineless Traversal competitive
  on arbitrary real-world web pages,
  the computationally expensive steps in spineless traversal
  have to be meticulously optimized
  to avoid allocation, reduce memory traffic,
  and optimize branch mispredictions.

\subsection{Pointer Compression and Custom Allocator}
Order maintenance objects have to be allocated every time
  new layout nodes are inserted;
  optimizing that allocation is essential.
We use a hand-written pool allocator for order maintenance objects;
  in fact, we use separate pools
  for high- and low-level list cells
  to enhance locality.
Since allocations are always for the same size,
  our custom allocator is significantly simpler than the system allocator.

Our pooled allocator, \texttt{OMPool},
  is shown in Figure~\ref{fig:allocator}.
It is paramaterized by \emph{two} types:
  the type of allocated object \texttt{T}
  and the index type \texttt{P} for pointers to allocated objects.
Crucially,
  \texttt{malloc} and \texttt{free} return and consume
  the pointer type \texttt{P} instead of raw pointers.
(And \texttt{addressof} function converts
  the pointer type \texttt{P} to a raw pointer.)
By making \texttt{P} a smaller integer type,
  like \texttt{uint32} or \texttt{uint16},
  order maintenance objects become smaller
  and more of them fit per cache line,
  which in turn improves throughput.
In our implementation, we use 32-bit pointers;
  this conveniently makes the total size
  of an order maintenance object 128 bits,
  meaning that it exactly fits in each cache line,
  with no order maintenance objects split across two.
By contrast, an experiment with 16-bit pointers
  actually increased runtime---because, we suspect,
  some order maintenance objects
  split across two cache lines,
  which introduced additional stalls.

The actual implementation of \texttt{OMPool} is standard;
  it stores a \texttt{pool} of memory as a \texttt{std::vector},
  in which we ensure sufficient capacity at startup.
Freed elements are placed in a separate \texttt{freed} vector,
  which is preferentially drawn from by \texttt{malloc}.
Because the objects are all the same size,
  there is no fragmentation and \texttt{malloc}/\texttt{free}
  are nearly instantaneous.
Moreover, since spineless traversal
  creates order maintenance objects in order,
  allocation patterns are extremely favorable,
  with temporally-close nodes placed nearby in memory.

\begin{figure}
\begin{verbatim}

template<
  typename T,             // Type of allocated object
  typename P=uint32_t     // Integer type for "pointers"
> struct OMPool {
  std::vector<T> pool;    // Fast allocation, maximize cache usage

  std::vector<P> freed;   // Rapid reuse minimizes cache churn
  
  // Implementation is straightforward
  T* addressof(P p) { return &(pool[p]); }
  P malloc();
  void free(P p) { freed.push_back(p); }
}
\end{verbatim}
\caption{A pooling allocator that focuses on reducing cache misses.}
\label{fig:allocator}
\end{figure}

\subsection{Branchless Order Maintenance Comparison}
Priority queue pushes and pops spend basically all of their time
  comparing order maintenance objects.
Order maintenance objects are small and,
  thanks to our allocator, typically stay in cache.
However, order maintenance object comparison has two cases
  (same or different second-level lists)
  and the bottleneck ends up being the pipeline stall
  induced by the conditional.
The branch predictor does not help much,
  because (thanks to the heap) the comparison is unpredictable.
We therefore implemented a branchless comparison function,
  relying on conditional move instructions instead;
  in a microbenchmark, this reduces comparison time
  to 5 cycles; Figure~\ref{fig:compare} shows the
  assembly implementation.


\begin{figure}

\begin{Verbatim}[formatcom=\color{gray}, commandchars={+[]}]
bool operator<(const _l2_node &l, const _l2_node &r) {
  Label lpl = l.parent->label, rpl = r.parent->label;
  Label ll = l.label, rl = r.label;
  uint64_t result;
  asm volatile(
  
    "+color[black][xor   %%rbx,  %%rbx]    +color[gray][\n"]
    "+color[black][cmp   %1,     %2]       +color[gray][\n"]
    "+color[black][seta  %%bl]             +color[gray][\n"]
    "+color[black][xor   %%rax,  %%rax]    +color[gray][\n"]
    "+color[black][cmp   %3,     %4]       +color[gray][\n"]
    "+color[black][seta  %%al]             +color[gray][\n"]
    "+color[red][cmove %%rbx,  %%rax]    +color[gray][\n"]
    
    : "=&a"(result) : "r"(ll), "r"(rl), "r"(lpl), "r"(rpl));
  return result;
}
\end{Verbatim}
\caption{
  The branchless comparison assembly code,
    with the conditional move instruction highlighted.
  Note that, while the assembly snippet
    is seven instructions long,
    the two \texttt{xor} instructions clear a register
    and are thus handled by the register renamer,
    and also the \texttt{cmp}/\texttt{seta} combination
    is fused on recent Intel/AMD CPUs.
  This means the assembly snippet itself
    typically takes three cycles to execute;
    even adding the label loads,
    comparison typically takes only five cycles.
}
\label{fig:compare}
\end{figure}

\subsection{Dependency Deduplication}

When synthesizing the dirty bit propagation code,
  we make sure to only dirty any given field once per field computation.
For example, if a field $N.V$ is used twice in an expression,
  or if two different paths $\mathsf{first}.V$ and $\mathsf{last}.V$
  reverse to the same field,
  the field is only dirtied once.
This deduplication is especially challenging in the case
  of $N?$ expressions,
  since whether or not a field is dirtied can depend on whether an element
  is the first/last/other child of its parent.
Moreover, the dirty bit is checked before pushing to the priority queue;
  this means any node field appears in the priority queue
  at most once, which keeps the queue small.

\subsection{Field packing}

To further shrink the size of the priority queue,
  we use a single dirty bit to cover multiple co-computed fields.
Instead of a dirty bit per field,
  we use two dirty bits per pass---%
  one for the pre-order-computed fields
  and one for post-order-computed fields---%
  and place these pass references,
  not individual fields, in the priority queue.
This further reduces the number of unique dirty bits
  set during dirty bit propagation
  and reduces the size of the priority queue
  and the number of order maintenance objects needed.
Field packing in this way is a common optimization in real browsers;
  the trade-off is that it reduces the complexity
  (and thus cache misses in) invalidation traversal
  at the cost of possibly more field recomputations
  (since the browser must now recompute \emph{all} covered fields
  when a dirty bit is set);
  the trade-off appears to be worth it.
Spineless Traversal can be applied to any field packing,
  whether finer- or coarser-grained,
  though coarser-grained packing makes
  the priority queue smaller,
  meaning that, with Spineless Traversal,
  browsers may opt for coarser-grained field packing
  then they do today.

\subsection{Field representation}

Recall that priority queue entries
  refer to either a subtree or a node,
  and also name a phase plus pre/post location.
We combine the subtree-or-node bit with the phase and pre/post location
  into a simple enum, which we store in a byte.
We use the enum as an index into a jump table,
  which allows us to efficiently execute the field recomputation
  that corresponds to a single priority queue entry.

\subsection{Attempted, Failed Optimizations}

A number of attempted optimizations
  failed to improve performance:
\begin{itemize}
\item A hybrid of Double Dirty Bit and Spineless Traversal,
    using a summary bit for subtree dirty bit propagation
    but the priority queue for more distant jumps.
  We were unable to make switching between the two modes
    efficient enough to be competitive.
\item A splay tree instead of a min-heap to improve priority queue locality.
  Resulted in a slowdown.
\item A red-black tree instead of a min-heap for the priority queue.
  Resulted in a slowdown.
\item 1-based array indices in the min-heap,
  which need fewer addressing instruction.
  Resulted in a slowdown.
\item 16-bit \texttt{OMPool} pointers, and 16-bit OM labels.
  No faster than 32 bits, and more rebalancings needed.
\item Pointer tagging---reduce priority queue node size
  by moving boolean bits into pointers. No faster.
\item Splitting OM \texttt{left}/\texttt{right} pointers
  from the \texttt{label} and \texttt{parent}/\texttt{children} pointers, to improve cache locality. Did not see improvement.
\item Deallocating OM objects when the corresponding tree node is deleted, to increase cache locality. Did not see improvement.
\end{itemize}